\documentclass[a4paper,12pt]{article}
\usepackage{cmap}
\usepackage[T2A]{fontenc}
\usepackage[utf8]{inputenc}
\usepackage[english,russian]{babel}
\usepackage{amsmath,amsfonts,amssymb,amsthm,mathtools,mathtext,nccmath,textcomp}
\usepackage[colorlinks,linkcolor=blue]{hyperref}
\usepackage{physics}
\usepackage{esvect}
\usepackage{fancyhdr}
\usepackage{icomma}
\usepackage{indentfirst}
\usepackage[labelsep=period,justification=centering]{caption}
\usepackage{upgreek}
\usepackage{lmodern}
\begin{document}
\thispagestyle{fancy}
\fancyhf{}
\fancyhead[C]{\S~ 5. Первообразная непрерывной функции}
\fancyhead[R]{357}

\noindent с помощью вычисления производной, что эта функция достигает локального минимума при $x_{0}=1/e$. При этом $f(1/e)=e^{-1/e}$, и это значение является ее наименьшим значением на сегменте [0,1]. Используя свойство б) настоящего пункта, получим, что $e^{-1/e}\leqslant\int\limits_{0}^{1}x^{x}dx\leqslant1$, а для числа $e^{-1/e}$ легко получить, что $e^{-1/e}=0,692\dots$. Заметим, что в этом случае значение интеграла нельзя определить через значения элементарных функций.

2) Если функция f(x) не является непрерывной, то формула среднего значения(**) может быть несправедливой. Рассмотрим функцию
\begin{equation*}
	f(x) = 
	\begin{cases}
		1/2, \ 0\leqslant x \leqslant 1/2, \\
		3/4, \ 1/2 < x \leqslant 1.
	\end{cases}
\end{equation*}

\noindent Тогда $\int\limits_{0}^{1}f(x)dx=5/8$. Значение 5/8 не принимается функцией f(x) ни в одной точке $\xi$ сегмента [0,1]. Следовательно, не существует числа $\xi\in$[0,1], для которого $\int\limits_{0}^{1}f(x)dx=f(\xi)$.

\begin{center}
\textbf{\S~5. ПЕРВООБРАЗНАЯ НЕПРЕРЫВНОЙ ФУНКЦИИ. ПРАВИЛА ИНТЕГРИРОВАНИЯ ФУНКЦИИ}
\end{center}

В предыдущих параграфах уже достаточно полно изучены свойства интеграла Римана. В частности, было показано, что, пользуясь определением интеграла, можно вычислить интеграл от некоторых простейших функций. Однако такое вычисление интеграла с помощью предельного перехода в интегральных суммах обладает рядом неудобств и приводит к значительным трудностям. Поэтому на повестку дня встает вопрос о простых правилах вычисления определенного интеграла Римана. Ниже нами будет дано одно из таких правил вычисления определенного интеграла, а именно будет доказана основная формула интегрального исчисления (формула Ньютона---Лейбница).

\textbf{1. Первообразная.}Рассмотрим интегрируемую на сегменте [a,b] функцию f(x). Пусть p принадлежит [a, b]. Тогда для любого x из [a, b] функция f(x) интегрируема на [p, x], и поэтому на сегменте [a, b] определена функция F(x)=$\int\limits_{p}^{x}f(t)dt$, которая называется интегралом с переменным верхним пределом. Аналогично определяется функция F(x) на интервале (a, b) при условии, что f(x) определена на интервале (a, b) и интегрируема на любом сегменте, принадлежащем этому интервалу.
\newpage
\thispagestyle{fancy}
\fancyhf{}
\fancyhead[C]{Гл. 9. Определенный интеграл Римана}
\fancyhead[L]{358}
Теорема 9.5. \textit{Если функция f(x) интегрируема на сегменте {\normalfont [}a, b{\normalfont ]}, p - любая точка этого сегмента, то производная функции F(x)=$\int\limits_{p}^{x}f(t)dt$ существует в каждой точке $x_0$ непрерывности подынтегральной функции, причем F'($x_0$)=f($x_0$)} \footnote{Если точка $x_0$ совпадает с одним из концов сегмента [a, b], то под производной в точке $x_0$ функции F(x) понимается левая или правая производная соответственно. Доказательство теоремы при этом не меняется.}.

\textit{Доказательство}. В силу непрерывности функции f(x) в точке $x_0$ для любого $\epsilon>0$ найдется такое $\delta>0$, что f($x_0$)---$\epsilon<f(x)<f(x_0)+\epsilon$, если $\abs{x-x_0}<\delta$. Для всех t из [$x_0, x$] выполняется неравенство $\abs{t-x_0}\leqslant\abs{x-x_0}<\delta$. Поэтому для всех таких t
\begin{equation*}
	f(x_0)-\epsilon\leqslant f(t)\leqslant f(x_0)+\epsilon.
\end{equation*}
Согласно свойству д) п. 2 \S~4 (независимо от знака разности $x-x_0$) получим из последних неравенств
\begin{equation*}
	f(x_0)-\epsilon\leqslant\frac{1}{x-x_0}\int\limits_{x_0}^{x}f(t)dt\leqslant f(x_0)+\epsilon\ \mbox{при}\ \abs{x-x_0}<\delta
\end{equation*}
(Значение $\mu$=$\frac{1}{x-x_0}\int\limits_{x_0}^{x}f(t)dt$ не меняется при перестановке чисел x и $x_0$, так как при этом одновременно меняется знак у величины $x-x_0$ и у интеграла $\int\limits_{x_0}^{x}f(t)dt$.). Но $\frac{1}{x-x_0}\int\limits_{x_0}^{x}f(t)dt=\frac{F(x)-F(x_0)}{x-x_0}$, следовательно, при $\abs{x-x_0}<\delta$
\begin{equation*}
	f(x_0)-\epsilon\leqslant\frac{F(x)-F(x_0)}{x-x_0}\leqslant f(x_0)+\epsilon,
\end{equation*}
т.е. F'(x) существует и равна $f(x_0)$. Теорема доказана.

Следствие. \textit{Любая непрерывная на сегменте [a, b] функция f(x) имеет на этом сегменте первообразную. Одной из первообразных является функция F(x)=$\int\limits_{a}^{x}f(t)dt$.}
	
Замечание 1. Теорема остается справедливой, если f(x) непрерывна на интервале (a, b). В этом случае в качестве нижнего предела надлежит взять любую точку p этого интервала. Все рассуждения сохраняются.
\newpage
\thispagestyle{fancy}
\fancyhf{}
\fancyhead[C]{\S~ 5. Первообразная непрерывной функции}
\fancyhead[R]{359}
Замечание 2. Можно рассматривать и функцию нижнего предела интеграла от f(x), т. е. функцию $\upphi(x)=\int\limits_{x}^{q}f(t)dt$. Для такой функции 
\begin{equation*}
	\upphi^{'}(x)=-f(x).
\end{equation*}

Замечание 3. Если функция f(x) интегрируема на любом сегменте, содержащемся в интервале (a, b), то интеграл с переменным верхнним пределом есть непрерывная на (a, b) функция верхнего предела.

Действительно, пусть $F(x) = \int\limits_{p}^{x}f(t)dt, p \in (a, b)$. Тогда 
\begin{equation*}
	\Updelta F = F(x+\Updelta x)-F(x)=\int\limits_{x}^{x+\Updelta x}f(t)dt=\mu\Updelta x,
\end{equation*}
где
\begin{equation*}
	\inf\limits_{x\in [x,x+\Updelta x]}f(x)\leqslant\mu\leqslant\sup\limits_{x\in [x,x+\Updelta x]}f(x)
\end{equation*}
по первой формуле среднего значения. Если функция f(x) интегрируема, то она ограничена, а поэтому для всех достаточно малых $\Updelta x$ ограничена и величина $\mu$, зависящая от $x$ и $\Updelta x$.
Более точно, $\inf\limits_{x\in [x,x+\Updelta x]}f(x)\leqslant\mu\leqslant\sup\limits_{x\in [x,x+\Updelta x]}f(x)$ \footnote{Здесь [c, d] - какой-либо фиксированный сегмент, принадлежащий интегралу (a, b) и такой, что $x \in [c, d], x+\Updelta x \in [c, d]$}. Поэтому $\Updelta F \to 0 \mbox{ при } \Updelta x \to 0$.

Замечание 4. Интегралы с переменным верхним (или нижним) пределом можно использовать для \textit{определения} новых функций, не выражающихся через элементарные функции.

Так, например, интеграл $\int\limits_{0}^{x}e^{-t^{2}}dt$, как уже отмечалось, называется интегралом Пуассона, интеграл $\int\limits_{0}^{x}\mfrac{dt}{\displaystyle\sqrt{(1-t^{2})(1-k^{2}t)}}$, $0<k<1$ называется эллиптическим интегралом, интеграл $\int\limits_{0}^{x}\frac{sint}t dt - $ интегральным синусом, $\int\limits_{1}^{x}\frac{costdt}t - $ интегральным косинусом и т. д.

\textbf{2. Основная формула интегрального исчисления.} Мы уже знаем из предыдущих рассмотрений, что любые две первообразные функции f(x), заданной на сегменте [a, b], отличаются на постоянную. Поэтому если $F(x) = \int\limits_{a}^{x}f(t)dt$, а $\upphi(x)$ - любая дру-
\newpage
\thispagestyle{fancy}
\fancyhf{}
\fancyhead[C]{Гл. 9. Определенный интеграл Римана}
\fancyhead[L]{360}
\noindent гая первообразная непрерывной функции f(x), то $\Upphi$(x)-F(x)=C=const, т. е. $\Upphi (x) = \int\limits_{a}^{x}f(t)dt+C$ (см. теорему 9.5). Положим в последней формуле сначала \textit{x=a}, а затем \textit{x=b}. Как мы условились (см. п. 1 предыдущего параграфа), $\int\limits_{a}^{a}f(t)dt=0$ для любой функции, принимающей конечное значение в точке $a$, поэтому $\Upphi(a)$=C, $\Upphi$(b)=$\int\limits_{a}^{a}f(x)dx+C$. Отсюда
\begin{equation*}
	\int\limits_{a}^{b}f(x)dx=\Upphi(b)-\Upphi(a),
\end{equation*}
и нами получена основная формула интегрального исчисления.

Сформулируем ее в виде теоремы.

Теорема (основная теорема интегрального исчисления). \textit{Для того чтобы вычислить определенный интеграл по сегменту [a, b] от непрерывной функции f(x), следует вычислить значение произвольной ее первообразной в точке b и в точке a и вычесть из первого значения второе.}

Теперь мы имеем правило вычисления определенного интеграла от широкого класса интегрируемых функций. Задача вычисления определенного интеграла свелась к задаче нахождения первообразной непрерывной функции. Естественно, что не для всякой функции найти первообразную просто. Мы уже неоднократно указывали функции, первообразные которых не выражаются через элементарные функции (см. например, п. 1 этого параграфа). В этом случае, естественно, возникает вопрос о приближенном вычислении определенных интегралов, о чем пойдет речь ниже.

Основную формулу интегрального исчисления часто записывают в форме $\int\limits_{a}^{b}f(x)dx=\Upphi(x)|_{a}^{b}$, где введено обозначение 
\begin{equation*}
	\Upphi(x)|_{a}^{b}=\Upphi(b)-\Upphi(a).
\end{equation*}

\textbf{3. Важные правила, позволяющие вычислять определенные интегралы.} При вычислении определенных интегралов очень часто используется правило замены переменной под знаком определенного интеграла.

\end{document}